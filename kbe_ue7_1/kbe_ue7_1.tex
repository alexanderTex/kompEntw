% /===========================================================================\
%
%   preamble
%
% \===========================================================================/

\documentclass[a4paper,letterpaper,10pt,ngerman]{scrartcl}

% /=================================================================\
%   	standard
% \=================================================================/
\usepackage{babel}
\usepackage[utf8]{inputenc}	% using umlauts = encoding
\usepackage[T1]{fontenc}

\usepackage{mdwlist}	% to shrink the spaces between the items into a list
\usepackage{blindtext}	% fill parts with blindtex
\usepackage{verbatim}	%u sing in multiline comments

\usepackage[german=guillemets]{csquotes}
\usepackage[
			backend=biber,
			style=alphabetic-verb,
			citestyle=alphabetic-verb
			]
			{biblatex}	% Einbinden von biblatex zur Literaturverwaltung.
\addbibresource{literaturelibrary/kbeLitLib.bib}	% Literatursammlung

\usepackage{amsmath}

% /=================================================================\
%   	commands
% \=================================================================/
\newcommand{\descitem}[1]{\textbf{#1}}	% style of description items


\renewcommand{\labelenumi}{\alph{enumi}}
% /===========================================================================\
%
%   init document
%
% \===========================================================================/

\begin{document}

% /===========================================================================\
%
%   title
%
% \===========================================================================/

% don't want date printed
\date{}

% make title bold and 14 pt font (Latex default is non-bold, 16 pt)
\title{\Large \bf Komponentenbasierte Entwicklung WiSe16/17 Übungsblatt 7: Expression Language und JSF-Lebenszyklus}

\author{
{\rm Alexander\ Lüdke}\\
MatrNr. 548965
\and
{\rm Nils Brandt}\\
MatrNr. 549906
}

\maketitle

\thispagestyle{empty}

% /===========================================================================\
%
%   main part
%
% \===========================================================================/
\begin{enumerate}
	\item EL2.1, was zuvor JSP2.1 gewesen war, schlägt eine Brücke zwischen den EL's aus JSP2.0 und JSF1.0, da sie diese in sich vereint.
	
	\item Sie kann in unterschiedlichen Phasen des Lifecycles vom Controller abgearbeitet werden. Die Expression \#{ myBean.myBeanPrope
rty } wird in der  Render-Response Phase des Lifecycles evaluiert und anschließend wird der Value als Outputtext gerendert.

	\item \#{myBean.myBeanProperty} => Wenn man den Wert ohne Metadaten angebene will.
    <h:outputText value="\#{myBean.myBeanProperty }" .../> => Hier können zu dem hinterlegten Wert Metadaten hinterlegt werden.
    
    \item Die p's in der Bean var werden von links nach rechts durchlaufen und p3, welcher ein Attribut oder eine Methode sein kann, wird aufgerufen.
    
    \item 
    \begin{itemize}
    	\item Methodenaufruf
    	\item Attributbearbeitung
    	\begin{itemize}
    		\item lesen
    		\item schreiben
    	\end{itemize}
    \end{itemize} 
    
    \item
    \begin{itemize*}
    	\item Implicit Object sind "Umgebungsvariablen", die zum aktuellen Status ausgewertet werden können.
    	\item Die folgenden werden zur Verfügung gestellt:
    	\begin{itemize*}
    		\item pageContext: The context for the JSP page. Provides access to various objects including:
    		\item servletContext: The context for the JSP page’s servlet and any web components contained in the same application. See Accessing the Web Context.
    		\item session: The session object for the client. See Maintaining Client State.
    		\item request: The request triggering the execution of the JSP page. See Getting Information from Requests.
    		\item response: The response returned by the JSP page. See Constructing Responses.
    	\end{itemize*}
    	\item In addition, several implicit objects are available that allow easy access to the following objects:
    	\begin{itemize*}
    		\item param: Maps a request parameter name to a single value
    		\item paramValues: Maps a request parameter name to an array of values
    		\item header: Maps a request header name to a single value
    		\item headerValues: Maps a request header name to an array of values
    		\item cookie: Maps a cookie name to a single cookie
    		\item initParam: Maps a context initialization parameter name to a single value
    	\end{itemize*}
    	\item Finally, there are objects that allow access to the various scoped variables described in Using Scope Objects.
		\begin{itemize*}
			\item pageScope: Maps page-scoped variable names to their values
			\item requestScope: Maps request-scoped variable names to their values
			\item sessionScope: Maps session-scoped variable names to their values
			\item applicationScope: Maps application-scoped variable names to their values
		\end{itemize*}
    \end{itemize*}
    
    \item Da die Werte mit NULL initialisiert werden, kann es dazu führen, dass wenn nach Werten gesucht wird, die nicht existieren, NULL als Rückgabewert ausgegeben werden.
    
    \item 
    \begin{itemize*}
    	\item Arithmetic: +, - (binary), *, / and div, \% and mod, - (unary)
    	\item Logical: $and, \&, or, \|, not, !$
    	\item Relational: ==, eq, !=, ne, <, lt, >, gt, <=, ge, >=, le. Comparisons can be made against other values, or against boolean, string, integer, or floating point literals.
    	\item Conditional: A ? B : C. Evaluate B or C, depending on the result of the evaluation of A.
    	\item (Empty: The empty operator is a prefix operation that can be used to determine whether a value is null or empty.)	
	\end{itemize*}     
    
    \item
    \begin{itemize}
    	\item Falsche Verwendung der Operatoren
    	\item Nutzung der reservieten Worte (or, not, eq, ...)
    \end{itemize}
    
    \item
    \begin{itemize}
    	\item HTML5 Friendly Markup, 
    	\item Resource Library Contracts and 
    	\item Faces Flows
    \end{itemize}
    
    \item Tomcat 7.0 uses the Jasper 2 JSP Engine to implement the JavaServer Pages 2.2 specification.
    

\end{enumerate}
%\newpage

% /===========================================================================\
%
% 	bibliography
%
% \===========================================================================/
\printbibliography

\end{document}







